\documentclass[a4paper,12pt]{article}

% --- Pacchetti ---
\usepackage[utf8]{inputenc}
\usepackage[T1]{fontenc}
\usepackage{parskip}
\usepackage{graphicx}
\usepackage{subcaption}
\usepackage{tabularx}
\usepackage{lmodern}
\usepackage[italian]{babel}
\usepackage{titlesec}
\usepackage{amsmath} 
\usepackage{geometry}
\usepackage{ifthen}
\usepackage{xstring}
\usepackage{helvet}
\usepackage{url}
\renewcommand{\familydefault}{\sfdefault}  % Sans serif font style

\geometry{top=2.5cm,bottom=2.5cm,left=2.5cm,right=2.5cm}

% --- Variabili personalizzabili ---
\newcommand{\tipoDiLaurea}{Magistrale}        % triennale, magistrale, ecc.
\newcommand{\corso}{Ingegneria Informatica - Ramo Automazione}
\newcommand{\materiaTesi}{Azionamenti ed Elettronica Industriale}
\newcommand{\thetitle}{Progettazione di un azionamento per motore in continua}
\newcommand{\theauthor}{Schettini Francesco}
\newcommand{\professore}{Prof. Ing. Luigi Rubino}               
\newcommand{\annoAccademico}{2024/2025}


% --- Inizio del documento ---
\begin{document}
\begin{titlepage}
\setlength{\parindent}{0cm}

% --- Header con logo e intestazioni ---
{\fontsize{10}{12}\selectfont
	\begin{tabular*}{\textwidth}{@{}l l @{\extracolsep{\fill}} r}
		\raisebox{-.4\height}{\includegraphics[height=1.8cm]{Immagini/Altre/V.jpg}} & 
		\parbox{5cm}{Università \\ degli Studi \\ della Campania \\ {\itshape Luigi Vanvitelli}} & 
		{\itshape Dipartimento di Ingegneria}
	\end{tabular*}
}

\vspace{5cm}

% --- Titolo e corso ---
{\centering
	{\fontsize{14}{18} \itshape Corso di Laurea \tipoDiLaurea{} in\par}
	{\fontsize{18}{22} \bfseries \corso \par}
	\vspace{1.5cm}
	{\fontsize{14}{18} \itshape Tesina di progetto in \materiaTesi \par}
	\vspace{0.5cm}
	{\fontsize{22}{26} \bfseries \thetitle \par}
}

\vfill

% --- Informazioni autore e relatore/i ---

\vspace{1cm}
\begin{center}
    \begin{tabular}{p{0.45\textwidth} p{0.45\textwidth}}
        \textbf{Candidato:} Schettini Francesco & \textbf{Relatore:} Ing. Luigi Rubino \\
        \textbf{Matricola:} A18000485 &  \\
    \end{tabular}
\end{center}
\rule{\textwidth}{0.4pt}

\vspace{1cm}

\begin{center}
	{\fontsize{10}{12}\bfseries A.A. 2024/2025}
\end{center}

\end{titlepage}


\renewcommand{\contentsname}{Indice}
\tableofcontents 

\pagebreak


\section{Introduzione}

Lo scopo del presente progetto è la progettazione e simulazione di un azionamento per motore in corrente continua (DC) spazzolato, da impiegare in un'applicazione reale scelta dallo studente.

Si farà uso della tecnica del controllo d'armatura. Tale metodo di controllo consente di agire direttamente sulla tensione applicata all'armatura del motore, imponendo la sua corrente, mantenendo costante il campo magnetico.

L’obiettivo finale è quello di modellare l’intero sistema meccatronico, selezionare un motore adeguato, progettare l’anello di controllo per la velocità e la corrente, simulare il comportamento dinamico del sistema in ambiente Simulink/Matlab e verificare il rispetto delle specifiche prestazionali fissate.




\pagebreak

\section{Applicazione, specifiche e scelta del motore}

\subsection{Specifiche di progetto}

Si vuole progettare un azionamento per lo scorrimento di un tapis roulant.

\begin{figure}[h!]
    \centering
    \includegraphics[scale=0.7]{Immagini/Altre/tapis.PNG}
    \caption{modello di tapis roulant}
    \label{fig: tapis}
\end{figure}


Le specifiche dell'applicazione sono le seguenti:

\begin{table}[h!]
    \centering
    \begin{tabular}{|l|l|}
    \hline
    \textbf{Caratteristica} & \textbf{Valore} \\ \hline
    Velocità regolabile & 0 -- 18 km/h \\ \hline
    Tempo massimo per raggiungere la velocità desiderata & 8 s \\ \hline
    Portata massima & 170 kg \\ \hline
    \end{tabular}
    \caption{Specifiche primarie del tapis roulant}
\end{table}

\subsection{Struttura meccanica}
Si prenda in esame la struttura meccanica descritta dal seguente CAD\footnote{\url{https://grabcad.com/library/conveyor-belt-123}.}.

\begin{figure}[h!]
    \centering
    \includegraphics[scale=0.35]{Immagini/Altre/CADtappeto.PNG}
    \caption{CAD struttura meccanica}
    \label{fig: CAD}
\end{figure}

\vspace{0.5cm}

La struttura meccanica analizzata è composta dai seguenti elementi principali:

\begin{itemize}
    \item Due rulli identici, di cui uno motorizzato e uno inerte;
    \item Un tappeto scorrevole, ancorato ai due rulli;
    \item Un motore a corrente continua spazzolato, dotato di sensori;
    \item Cuscinetti, utilizzati per garantire uno scorrimento favorevole del tappeto.
\end{itemize}

\vspace{0.5cm}

Le dimensioni e caratteristiche geometriche dei componenti sono riassunte nella Tabella~\ref{tab:dati}.

\begin{table}[h!]
    \centering
    \begin{tabular}{|l|c|}
        \hline
        \textbf{Componente} & \textbf{Valore} \\
        \hline
        Dimensioni tappeto & $120\ \text{cm} \times 50\ \text{cm}$ \\
        Massa tappeto & $5\ \text{kg}$ \\
        Diametro rulli & $5\ \text{cm}$ \\
        Massa di ciascun rullo & $6\ \text{kg}$ \\
        \hline
    \end{tabular}
    \caption{Parametri fisici della struttura}
    \label{tab:dati}
\end{table}
\vspace{0.5cm}
 
A partire da questi dati è possibile ricavare i corrispondenti \textbf{momenti d'inerzia} dei componenti per l'analisi dinamica del sistema\footnote{Si è usato per il tappeto l'equivalente rotazionale di un'inerzia traslatoria}.

\vspace{0.5cm}

\[
    J_{\text{rullo}} = \frac{1}{2} m_{\text{rullo}} r_{\text{rotazionale}}^2
\]

\[
    J_{\text{tappeto}} = m_{\text{tappeto}} r_{\text{rotazionale}}^2
\]

\[
    J_{\text{macchina}} = 2 J_{\text{rullo}} + J_{\text{tappeto}} = 0.0069\, kg\cdot m^2
\]

\vspace{0.5cm}

Ricordiamo che montato al rullo di trazione, dunque all'albero, ci sarà il nostro motore. Pertanto, nel conto totale dell'inerzia servirà anche la sua inerzia.


\subsection{Scelta del motore}

Il settore applicativo ci fa ipotizzare di aver bisogno di un motore da almeno 1~kW. In seguito, verificheremo che il motore scelto sia adatto alla nostra applicazione.

\vspace{0.5em} 

Scegliamo il motore modello L80B14 a magneti permanenti della casa produttrice Elip Tagliente, di cui presentiamo in Tabella~\ref{tab: datiTargaMotore} i dati di targa.

\vspace{0.3em} 

\begin{table}[h!]
    \centering
    \begin{tabular}{|l|l|}
    \hline
    \textbf{Caratteristica} & \textbf{Valore} \\ \hline
    Tensione nominale & 24 V \\ \hline
    Corrente nominale & 54 A \\ \hline
    Potenza nominale & 1000 W (meccanica) \\ \hline
    Velocità nominale & 3000 RPM \(\rightarrow\) 314 rad/s \\ \hline
    Coppia nominale & 3.18 Nm \\ \hline
    Peso & 6.15 kg \\ \hline
    Diametro rotore & 11 cm \\ \hline
    \end{tabular}
    \caption{Dati di targa del motore}
    \label{tab: datiTargaMotore}
\end{table}

\vspace{0.5em} 

Si osservi che sul sito di vendita\footnote{\url{https://www.shop.eliptagliente.it/prodotto/motore-elettrico-24vdc-1000w-3000rpm-l80b14/}.} sono indicate le sue principali applicazioni e si può constatare che una di queste è proprio l'impiego per tapis roulant.

\begin{figure}[h!]
    \centering
    \begin{subfigure}[b]{0.45\textwidth}
        \centering
        \includegraphics[width=\linewidth]{Immagini/Altre/motoreElipTagliente.PNG}
        \caption{Scheda tecnica}
        \label{fig:motore1}
    \end{subfigure}
    \hfill
    \begin{subfigure}[b]{0.45\textwidth}
        \centering
        \includegraphics[width=\linewidth]{Immagini/Altre/motoreElipTaglienteDisegno.PNG}
        \caption{Disegno in sezione}
        \label{fig:motore2}
    \end{subfigure}
    \caption{Motore Elip Tagliente}
    \label{fig:Motore1+2}
\end{figure}

\vspace{0.5em}


Dai dati di targa forniti è stato possibile ricavarne altri

\[
    P_{a, \text{elettr}} = V_a \cdot I_a = 1296\,W
\]

\[
    J_{\text{motore}} = \frac{1}{2} m_{\text{motore}} \left(\frac{d_{\text{motore}}}{2}\right)^2 = 0.009 \, kg \cdot m^2
\]

\vspace{0.5em}

\subsection{Specifiche tecniche per la scelta del motore}

Dalle specifiche primarie è possibile ricavare le seguenti grandezze utili per il dimensionamento del motore:

\begin{itemize}
    \item Velocità rotazionale massima necessaria
    \item Coppia massima necessaria
    \item Potenza richiesta
\end{itemize}

\vspace{0.5em}

Dalle specifiche sulla velocità e la struttura dei rulli, si ricava la velocità angolare massima\footnote{usando la relazione \( v = \omega r \):}

\[
    \omega_{\text{max}} = \frac{v_{\text{tapis, max}} / 3.6}{r_{\text{rotazionale}}} = 200 \,rad/s
\]

\vspace{0.5em}
 
Dalle specifiche sull'accelerazione, sappiamo che il sistema deve raggiungere la velocità massima in un tempo prestabilito, trasportando un carico pari alla portata massima. Usiamo una massa campione pari a \(1.25 \cdot p\), che offre due vantaggi:

\begin{itemize}
    \item un margine di robustezza ulteriore rispetto al semplice sovradimensionamento della potenza;
    \item una modellazione più realistica dei picchi di massa percepita durante la corsa, in particolare nella fase di atterraggio del piede.
\end{itemize}

\vspace{0.5em}

Calcoliamo l'inerzia rotazionale del carico\footnote{Così come per il nastro, è stata considerata l'inerzia rotazionale equivalente, pur essendo un'inerzia traslatoria}:
\[
    m_{\text{carico}} = \text{sovradimensionamento}_{\text{carico}} \cdot       p
\]
\[
    J_{\text{carico}} = m_{\text{carico}} \cdot r_{\text{rotazionale}}^2= 0.133 \, kg \cdot m^2
\]

\vspace{0.5em}

Così è possibile calcolare l'inerzia totale vista sull'albero motore:

\vspace{0.3em}

\[
    J_{\text{tot}} = J_{\text{macchina}} + J_{\text{motore}} + J_{\text{carico}}
\]

\vspace{0.5em}

Sapendo di dover raggiungere la velocità massima richiesta in un tempo fissato da specifica, si può calcolare l'accelerazione massima da imprimere:
\vspace{0.5em}
\[
    \alpha_{\text{max}} = \frac{\omega_{\text{max}}}{\Delta t_{\text{acc,max}}}=25 \, rad/s^2
\]

\vspace{0.5em}

Dall'accelerazione massima, è possibile trovare la coppia massima che il motore deve generare:
\[
    C_{\text{max}} = J_{\text{tot}} \cdot \alpha_{\text{max}} = 3.72 \,Nm
\]

\vspace{0.5em}
Infine, si può calcolare la potenza meccanica necessaria a coprire il set applicativo:

\[
    P_{\text{richiesta}} = C_{\text{max}} \cdot \omega_{\text{max}} = 745\,W
\]

\vspace{0.5cm}

\begin{table}[h!]
    \centering
    \begin{tabular}{|l|c|}
    \hline
    \textbf{Grandezza} & \textbf{Valore richiesto} \\ \hline
    Velocità angolare massima  & 200 rad/s\quad \\ \hline
    Coppia massima & 3.72 Nm\quad \\ \hline
    Potenza massima & 745 W\quad \\ \hline
    \end{tabular}
    \caption{Specifiche richieste al motore}
\end{table}


\subsection{Validazione della scelta e introduzione del riduttore}

Pur ipotizzando un 10\% di perdite meccaniche (per attrito), la potenza del motore scelto risulta comunque ampiamente sufficiente per la nostra applicazione:

\[
    P_{a, \text{mecc,effettivo}} = 0.9 P_{a, \text{mecc}}=900\,W
\]

\vspace{0.5cm}

Tuttavia, si osservi anche

\begin{itemize}
    \item La coppia massima richiesta non è soddisfatta dalla coppia nominale del motore
    \item La velocità nominale del motore è ampiamente superiore a quella necessaria
\end{itemize}

\vspace{0.5cm}

Si decide dunque di adottare un riduttore con rapporto di trasformazione \( R_{\text{rid}} \), scelto in modo tale da ottenere, a valle del riduttore:

\[
    \omega_{valle,rid} = \omega_{max}
\]

\vspace{0.5cm}
A tal fine è necessario avere un rapporto di riduzione 

\[
    R_{rid} = \omega_{nom} / \omega_{max}  = 1.57
\]

\vspace{0.5cm}

Scegliamo il riduttore della SEW ad uno stadio\footnote{\url{https://www.sew-eurodrive.it/prodotti/riduttori/riduttori_standard/riduttori_ad_ingranaggi_r/riduttori_ad_ingranaggi_r.html}.}

\begin{figure}[h!]
    \centering
    \includegraphics[scale=0.6]{Immagini/Altre/riduttore.PNG}
    \caption{Riduttore SEW ad uno stadio}
    \label{fig: riduttore SEW}
\end{figure}

\vspace{0.5cm}

L'inerzia per questo tipo di riduttori è dell'ordine dei $10^{-3}/10^{-4}\,kg \cdot m^2$, dunque trascurabile rispetto alle nostre grandezze inerziali.

\vspace{0.5cm}

Inoltre, segnaliamo che in genere tali riduttori causano una perdita di efficienza introducendo perdite meccaniche. Tuttavia, noi lo supporremo ideale.

\vspace{0.5cm}
Con l'introduzione del riduttore, così sono cambiate le caratteristiche nominali 

\begin{table}[h!]
    \centering
    \begin{tabular}{|l|c|l|c|}
    \hline
    \textbf{Velocità richiesta} & 200 rad/s & \textbf{Velocità nominale ottenuta} & 200 rad/s \\ \hline
    \textbf{Coppia richiesta}  &  3.72 Nm  & \textbf{Coppia nominale ottenuta}   & 4.99 Nm              \\ \hline
    \end{tabular}
    \caption{Confronto tra grandezze richieste e ottenute}
\end{table}

\vspace{0.3cm}

Pertanto, la coppia motore-riduttore così scelta rientra nelle specifiche per la mia applicazione.

\vspace{0.5cm}

NOTA: è stato preferito avere un range largo in coppia piuttosto che in velocità, in quanto la velocità 200 rad/s è la massima da specifica della mia applicazione e non potrò mai averne di superiori.

Al contempo, poter generare una coppia ancor più alta di quella già ipotizzata per il caso peggiore, rende ancor più robusta l'applicazione e fornisce margini di miglioramento futuro.

\newpage

\section{Modello del motore}

\subsection{Parametrizzazione}

Abbiamo bisogno di stimare o ricavare alcuni parametri fondamentali del motore:

\begin{itemize}
    \item Resistenza d'armatura \( R_a \)
    \item Induttanza d'armatura \( L_a \)
    \item Coefficiente di attrito viscoso \( \beta \)
    \item Costanti del motore: \( k_t \) e \( k_v \)
\end{itemize}

\vspace{0.5cm}
La Resistenza d'armatura si può calcolare a partire dalle perdite elettriche:

\[
    P_{\text{perdite,elett}} = P_{a, \text{elett}} - P_{a, \text{mecc}} = 296\,W
\]

\[
    R_a = \frac{P_{\text{perdite, elett}}}{I_a^2}=0.10\,\Omega
\]
\vspace{0.5cm}

L'induttanza d'armatura non è possibile calcolarla direttamente dalle specifiche, sarebbe necessario effettuare prove sul motore. Pertanto, scegliamo un valore tipico per questa classe di motori:

\[
    L_a = 1 \cdot 10^{-4} \ \text{H}
\]

\vspace{0.5cm}

Il coeff. di attrito viscoso può essere calcolato a partire dalle perdite meccaniche (stimate al 10\% della potenza meccanica):

\[
    \beta = \frac{P_{\text{perdite,mecc}}}{\omega_{a, \text{motore}}^2}=1 \cdot 10^{-3}\,kg/ms
\]

\vspace{0.5cm}
Per le costanti macchina, sappiamo dalla teoria che:

\[
    k_t = \frac{C_a}{I_a}=0.059
\]

\[
    k_v = \frac{V_a - R_a \cdot \omega_{a, \text{motore}} \cdot \beta / k_t}{\omega_{a, \text{motore}}}=0.075
\]

\vspace{0.5cm}
L'inerzia del motore, invece è stata già calcolata nei precedenti paragrafi.

Notifichiamo che nel nostro modello di motore, assumeremo per semplicità che abbia inerzia quella della macchina intera.

Infatti, la macchina è una componente fissa montata sull'albero motore. Preferiamo piuttosto modellare come carico tutto ciò che è variabile (come ad esempio persone che salgono sul tapis).

\[
    J = J_{\text{macchina}} + J_{\text{motore}}=0.0078\,kg \cdot m^2
\]

\vspace{0.5cm}

Nelle Tabelle~\ref{tab: parametriFisiciMotore}-\ref{tab: costantiMacchinaMotore} sono riassunti i parametri del motore.

\vspace{0.5cm}

\begin{table}[h!]
    \centering
    \begin{tabular}{|l|c|}
    \hline
    \textbf{Parametro} & \textbf{Valore} \\ \hline
    Resistenza d'armatura  & 0.10 $\Omega$  \\
    Induttanza d'armatura &  $1 \cdot 10^{-4}$ \ \text{H}\\
    Inerzia (totale)  & $0.0078\,kg \cdot m^2$ \\
    Coefficiente attrito viscoso  & $1 \cdot 10^{-3}\,kg/ms$ \\ \hline
    \end{tabular}
    \caption{Parametri del motore}
    \label{tab: parametriFisiciMotore}
\end{table}


\begin{table}[h!]
    \centering
    \begin{tabular}{|l|c|}
    \hline
    \textbf{Parametro} & \textbf{Valore} \\ \hline
    $k_t$ & 0.059  \\
    $k_v$ & 0.075\\ \hline
    \end{tabular}
    \caption{Costanti macchina del motore}
    \label{tab: costantiMacchinaMotore}
\end{table}


\subsection{Schema a blocchi del motore}


Dalla teoria, il motore è descritto dal seguente schema a blocchi:

\begin{figure}[h!]
    \centering
    \includegraphics[scale=0.7]{Immagini/Altre/motoreSchemaBlocchi.PNG}
    \caption{Schema a blocchi del motore}
    \label{fig: schema a blocchi}
\end{figure}

\vspace{0.3cm}

Dai parametri ricavati nella sezione precedente, è stato possibile determinare le funzioni di trasferimento relative al dominio elettrico e meccanico:

\vspace{0.5cm}

$$
P_{\text{elet}}(s) = \frac{1}{L_a s + R_a} \quad\quad
P_{\text{mec}}(s) = \frac{1}{J_{\text{tot}} s + \beta}
$$

\vspace{0.5cm}

Il modello del motore è stato implementato successivamente in Simulink, come riportato in Figura~\ref{fig:modello_simulink}.


\begin{figure}[h!]
\centering
    \includegraphics[scale=0.6]{Immagini/Altre/motoreSimulink.PNG} 
    \caption{Modello motore in corrente continua in Simulink}
    \label{fig:modello_simulink}
\end{figure}
\vspace{0.5cm}

Scegliendo i parametri calcolati in precedenza, eseguiamo la simulazione del modello applicando una tensione nominale $V_{nom}=24~V$. In figura \ref{fig: ciclo aperto} è riportato l'andamento della velocità angolare $\omega$.

\vspace{0.3cm}

\begin{figure}[h!]
\centering
    \includegraphics[scale=0.6]{Immagini/PlotMatlab/rispostaMotoreCicloAperto.png} 
    \caption{Risposta a gradino del motore a ciclo aperto}
    \label{fig: ciclo aperto}
\end{figure}

\vspace{0.5cm}

Il valore di regime risulta essere 

\[
    \omega_{regime}= 314\,rad/s=\omega_a
\]
\vspace{0.3cm}

Il fatto che la velocità angolare a regime coincida con la velocità nominale del motore, secondo i dati di targa, certifica la bontà dello schema implementativo.

NOTA: non è stato tenuto conto del riduttore, che sarà modellato esternamente nello schema a blocchi finale.


\subsection{Blocchi aggiuntivi}


Per predisporre il motore al controllo, introduciamo i seguenti componenti:

\begin{itemize}
    \item Amplificatore
    \item Dinamo tachimetrica
    \item Saturatore in corrente
    \item Riduttore
\end{itemize}

\subsubsection{Amplificatore di tensione}

Immaginiamo di impiegare per il controllo un microcontrollore, che può generare tensioni tra $Gnd = 0$ e $HighVoltage = 5V$. Avremo bisogno di un amplificatore con guadagno $V_{nom}/HighVoltage$ per arrivare alle tensioni del motore.

Lo supporremo lineare con guadagno:

\vspace{0.3cm}

\[
    K_{amp} = \frac{V_{nom}}{HighVoltage} = \frac{24V}{5V}=4.8
\]


\vspace{0.5cm}

Nello schema simulink, mostrato in Figura~\ref{fig:simulink_controllo_ampl_sat} è modellato come un puro guadagno con limite in saturazione.


\subsubsection{Dinamo tachimetrica}

Al fine di realizzare un controllo in velocità a ciclo chiuso, è necessario misurare tale grandezza e retroazionarla, in particolare, convertendola in tensione. A tale scopo viene utilizzato come trasduttore di velocità una dinamo tachimetrica.

\vspace{0.5cm}

Utilizzeremo la seguente dinamo, modello DT160 della casa produttrice Hohner.


\begin{figure}[h!]
\centering
    \includegraphics[scale=0.5]{Immagini/Altre/dinamo.PNG}
    \caption{Dinamo DT160 - Hohner SRL}
    \label{fig:dinamo}
\end{figure}

\vspace{0.5cm}

Dal suo datasheet\footnote{\url{https://hohner.it/datasheet/itaing/vari/Cat_DT1.pdf?x18985}} è possibile estrarre alcuni suoi utili dati di targa, riportati in Tabella~\ref{tab:targa_dinamo}.

\begin{table}
    \centering
    \begin{tabular}{|l|c|}
    \hline
    \textbf{Parametro} & \textbf{Valore} \\
    \hline
    Tensione d'uscita a velocità fissata & 60 V a 1000 RPM \\
    \hline
    Numero lamelle & 33 \\
    \hline
    Ripple Peak-to-Peak massimo & 1\% dell'output \\
    \hline
    \end{tabular}
    \caption{Dati di targa della dinamo}
    \label{tab:targa_dinamo}
\end{table}

Dalla prima entry della tabella è possibile ricavare la costante tachimetrica.

\[
    k_{dt} = \frac{V_{nom,\text{din}}}{\omega_{nom,\text{din}}} =0.57 \,V/(rad/s)
\]

Tale segnale dev'essere compatibile con l'ingresso del microcontrollore, ovvero [Gnd, HighVoltage]. Pertanto, la sua uscita dev'essere mappata con un de-amplificatore.

\[
    k_{\text{deamp}} = \frac{\text{High\_voltage}}{\omega_{\text{nom}} \cdot k_{dt}}
\]

\vspace{0.3cm}

In questo modo, alla velocità massima del motore corrisponderà una tensione in retroazione di $HighVoltage = 5V$ e a velocità nulla una tensione nulla.

\vspace{0.5cm}
 
Sceglieremo durante il progetto di fare simulazioni con una dinamo ideale o reale.

Per la dinamo reale modelleremo la componente rumore con la sua sola armonica fondamentale, che ha: 

\begin{itemize}
    \item Frequenza = Nlamelle $\cdot$ w (velocità di rotazione del motore)
    \item Ampiezza = Ripple di tensione ($1\%$ dell'uscita ideale)
\end{itemize}

\vspace{0.5cm}

In Simulink la dinamo è implementata come in Figura~\ref{fig:simulink_dinamo} ed è inserita nello schema totale di controllo come in Figura~\ref{fig:simulink_controllo_dinamo}.

\begin{figure}[h!]
\centering
    \includegraphics[scale=0.5]{Immagini/Altre/dinamoSimulink.PNG}
    \caption{Modello simulink della dinamo}
    \label{fig:simulink_dinamo}
\end{figure}

\begin{figure}[h!]
\centering
    \includegraphics[scale=0.6]{Immagini/Altre/dinamoNelloSchemaSimulink.PNG}
    \caption{Dinamo nello schema di controllo}
    \label{fig:simulink_controllo_dinamo}
\end{figure}


\vspace{0.5cm}

\subsubsection{Saturatore clamping in corrente}

In fase di avvio del motore (o altri casi di discontinuità di funzionamento), la corrente di armatura raggiunge picchi molto alti e supera il valore nominale. In queste condizioni si danneggia il motore.

Per questo motivo è stato necessario dotare il motore di un limitatore di corrente.

\vspace{0.3cm}

Il limitatore scelto di tipo "clamping" è costituito da un blocco soglia (DEAD ZONE) e da un guadagno. 

Il blocco soglia fornisce un'uscita nulla per valori di ingresso tra le due soglie, mentre se l’ingresso assume valori al di fuori, restituisce in uscita la quantità in eccesso/difetto.
Le soglie sono state fissate al valore della corrente di armatura nominale $I_{a} = 54~A$, consentito dal motore.

Il guadagno $k_{compenso}$ è stato trovato con taratura manuale monitorando la corrente a tensione d'ingresso massima, finché non è risultata completamente satura attorno a Ia, consentendo valori al più del $20\%$. 

Si è usato per la taratura, nello schema a ciclo chiuso, un regolatore fittizio a guadagno unitario e la dinamo supposta ideale.

\vspace{0.5cm}

L'implementazione Simulink è mostrata in Figura ~\ref{fig:simulink_controllo_ampl_sat}.

\vspace{0.5cm}

\begin{figure}[h!]
\centering
    \includegraphics[scale=0.7]{Immagini/Altre/amplificatoreSaturatoreSimulink.PNG}
    \caption{Amplificatore e saturatore nello schema di controllo}
    \label{fig:simulink_controllo_ampl_sat}
\end{figure}

\subsubsection{Riduttore}

Dell'utilizzo del riduttore, al fine di adattare il motore alle specifiche in coppia e velocità della nostra applicazione, abbiamo già discusso.

Mostriamo la sua modellazione Simulink in Figura ~\ref{fig:simulink_controllo_riduttore}.

\vspace{0.5cm}

\begin{figure}[h!]
\centering
    \includegraphics[scale=0.6]{Immagini/Altre/riduttoreSimulink.PNG}
    \caption{Riduttore nello schema di controllo}
    \label{fig:simulink_controllo_riduttore}
\end{figure}

\vspace{0.5cm}

In pratica viene modellato con una serie di guadagni che scalano opportunamente le grandezze del motore.

\[
C_{\text{out}} = C \cdot R_{\text{rid}} \quad ; \quad
\omega_{\text{out}} = \frac{\omega}{R_{\text{rid}}} \quad ; \quad
Q_{\ell,in} = \frac{Q_{\ell}}{R_{\text{rid}}}
\]

\vspace{0.5cm}
NOTA: è stata scalata anche la coppia di carico, in quanto questa è applicata direttamente all'albero motore sul quale si agganciano direttamente gli ingranaggi. Questi poi trasferiscono la coppia scalata al motore.


\subsection{Schema finale pre-regolatore}

\begin{figure}[h!]
\centering
    \includegraphics[scale=0.5]{Immagini/Altre/preRegolatoreSimulink.PNG}
    \caption{Schema finale prima del regolatore}
    \label{fig:simulink_pre_regolatore}
\end{figure}


\newpage

\section{Progetto del controllore}
L'obiettivo di questo progetto è quello di controllare in velocità il sistema. A tal scopo bisogna trovare un regolatore che, chiudendo il loop in Figura~\ref{fig:simulink_pre_regolatore}, garantisca ottime caratteristiche a ciclo chiuso come tempo di risposta breve, sovraelongazione limitata e buoni margini di robustezza.

NOTA: in questa specifica applicazione l'asservimento in posizione non è necessario, dunque non viene realizzato.

\subsection{Regolatore PI}
\label{specifiche2}
Per il controllo di velocità del motore utilizziamo un regolatore PI.

\[
    C(s)=K_p + \frac{K_i}{s}
\]


La sola azione proporzionale infatti non basta, perché per seguire il riferimento a gradino abbiamo bisogno di un guadagno molto alto. Considerando la dinamo rumorosa (che è il modello più vicino alla realtà) viene amplificato anche il rumore stesso.

Questo causa un ripple di corrente molto alto, che usura il motore e ne riduce notevolmente il tempo di vita.

Allora s'introduce il termine integrale che garantisce l'astatismo a gradino e allo stesso tempo reietta il rumore, operando da filtro.


\begin{figure}[h!]
\centering
    \includegraphics[scale=0.7]{Immagini/Altre/PISimulink.PNG}
    \caption{Schema PI Simulink}
    \label{fig:simulink_PI}
\end{figure}




\subsubsection{Taratura del parametro $K_p$}

Spegniamo la parte integrativa.

Usiamo il modello di dinamo tachimetrica reale, che introduce il rumore responsabile di oscillazioni sulla corrente. Conduciamo simulazioni fino a soddisfare le seguenti condizioni:

\begin{itemize}
    \item Il ripple della corrente deve essere inferiore all'1\% di $I_{nom}$ (specifica di progetto);
    \item La corrente deve rimanere entro il 10\% di $I_{nom}$ (per pochi secondi).
\end{itemize}

Per garantire la robustezza, consideriamo il caso peggiore: durante le simulazioni forniamo al motore la massima tensione d'ingresso, ovvero \texttt{HighVoltage}.

\vspace{0.5cm}

Sono state condotte delle simulazioni con uno script del tutto automatizzato che ci hanno restituito i seguenti valori:

\begin{figure}[h!]
\centering
\begin{minipage}{0.60\textwidth}
    \centering
    \includegraphics[width=\linewidth]{Immagini/PlotMatlab/confrontoCorrentiKp.png}
    \caption{Ia al variare di $K_p$}
    \label{fig:andamento_kp}
\end{minipage}
\hfill
\begin{minipage}{0.38\textwidth}
    \centering
    \small
    \begin{tabular}{|c|c|c|}
        \hline
        \textbf{$K_p$} & \textbf{Ripple (A)} & \textbf{I Max (A)} \\
        \hline
        625.00 & 19.12 & 85.23 \\
        312.50 & 16.63 & 69.61 \\
        156.25 & 8.02  & 61.80 \\
        78.12  & 2.43  & 57.89 \\
        39.06  & 14.54 & 55.94 \\
        19.53  & 4.60  & 54.97 \\
        9.77   & 4.72  & 54.48 \\
        4.88   & 2.21  & 54.23 \\
        2.44   & 1.05  & 54.11 \\
        1.22   & 0.52  & 54.05 \\
        \hline
    \end{tabular}
    \captionof{table}{Simulazioni}
    \label{tab:iterazioni_kp}
\end{minipage}
\end{figure}


Per il soddisfacimento delle specifiche si è scelto quindi:

\[
    K_p = 1.22
\]

In figura è riportato l'andamento della corrente di armatura e della velocità con azione (solo) proporzionale trovata.

\begin{figure}[h!]
    \centering
    \includegraphics[scale = 0.6]{Immagini/PlotMatlab/controlloSoloProporzionale.png}
    \caption{Correnteìe velocità con Kp=1.22}
    \label{fig:kpscelto}
\end{figure}

\vspace{0.5cm}

Si osserva che la corrente ha una zona di saturazione approssimabile a piatta e il ripple di corrente è così piccolo da essere impercettibile sul grafico.
In tali condizioni, il rumore di misura non induce alcuna conseguenza.

Riguardo invece l'inseguimento a gradino, come ci aspettavamo dalla teoria dei sistemi, il riferimento viene seguito con errore a regime finito, che però risulta essere notevole in quanto il guadagno è molto piccolo.


\subsubsection{Taratura del parametro $K_i$}

Studiamo la risposta a gradino del sistema al variare di Ki.  Vogliamo:

\begin{itemize}
    \item sovraelongazione della riposta a gradino in velocità, del massimo $20\%$
\end{itemize}

Stavolta diamo in ingresso un piccolo step, preoccupandoci che non mandi in saturazione il sistema.

Non avendo problemi sul disturbo con la parte integrativa e avendo già tarato appositamente la parte proporzionale, userò il modello di dinamo ideale, ignorando il rumore.

\vspace{0.5cm}

Sono state eseguite delle simulazioni con uno script del tutto automatizzato che ci ha restituito i seguenti valori:

\begin{figure}[h!]
\centering
\begin{minipage}{0.6\textwidth}
    \centering
    \includegraphics[width=\linewidth]{Immagini/PlotMatlab/confrontoUsciteKi.png} 
    \caption{$w_{out}$ al variare di ki}
    \label{fig:sovraelongazione_ki}
\end{minipage}
\hfill
\begin{minipage}{0.38\textwidth}
    \centering
    \small
    \begin{tabular}{|c|c|}
        \hline
        \textbf{$K_i$} & \textbf{Sovraelongazione (\%)} \\
        \hline
        1.00  & 0.0000 \\
        1.30  & 0.0000 \\
        1.69  & 0.0000 \\
        2.20  & 0.0000 \\
        2.86  & 0.0000 \\
        3.71  & 0.0021 \\
        4.83  & 0.0263 \\
        6.27  & 0.0641 \\
        8.16  & 0.1134 \\
        10.60 & 0.1902 \\
        13.79 & 0.3014 \\
        \hline
    \end{tabular}
    \captionof{table}{Simulazioni}
    \label{tab:sovraelongazione_ki}
\end{minipage}
\end{figure}


Per il soddisfacimento delle specifiche si è scelto quindi:

\[
    K_i = 10.60
\]

\begin{figure}[h!]
    \centering
    \includegraphics[scale=0.6]{Immagini/PlotMatlab/rispostaGradinoDopoTaratura.png}
    \caption{Risposta a gradino con kp=1.22 e ki=10.60}
    \label{fig:kiscelto}
\end{figure}

\vspace{0.5cm}

Il grafico delle risposte a gradino in funzione di $K_i$ evidenzia che all’aumentare di $K_i$:
\begin{itemize}
    \item aumenta la sovraelongazione $S\%$;
    \item diminuisce il tempo di risposta $T_a$.
\end{itemize}

Scegliendo un valore di $K_i$ che garantisce una sovraelongazione inferiore al 20\%, si è ottenuto un buon compromesso tra rapidità e stabilità della risposta.


\vspace{0.5cm}
Si osserva inoltre dalla Figura~\ref{fig:kiscelto} che, con i valori scelti, il sistema è in grado di seguire correttamente il riferimento con errore nullo a regime, come previsto dalla t. dei sistemi.


\subsection{Fenomeno del Wind-up: correzione}

Immaginiamo di voler far generare il controllo dallo stesso microcontrollore usato per generare il riferimento.
Questo può generare il segnale di controllo solo in un range limitato [Gnd, HighVoltage], quindi per valori superiori (o inferiori), il controllo sarà saturato.

Le simulazioni\footnote{Qui non riportate} mostrano che l'aggiunta del blocco di saturazione in simulink(vedi Figura ~\ref{fig: regolatoreAntiWU})  non degrada le prestazioni, in quanto a saturare era già il blocco amplificatore.

Si manifesta però il fenomeno del wind-up sul controllo!

\vspace{0.5cm}

Risolveremo tale fenomeno usando il seguente schema di de-saturazione:
\begin{figure}[h!]
    \centering
    \includegraphics[scale=0.7]{Immagini/Altre/antiwindupSchema.PNG}
    \caption{Schema anti wind-up}
    \label{fig:antiWindup}
\end{figure}

\vspace{0.5cm}

L'implementazione Simulink di tale schema è mostrata in Figura~\ref{fig: regolatoreAntiWU}

\begin{figure}[h!]
    \centering
    \includegraphics[scale = 0.6]{Immagini/Altre/regolatoreDesaturatoSimulink.PNG}
    \caption{Regolatore con correzione anti wind-up}
    \label{fig: regolatoreAntiWU}
\end{figure}

\vspace{0.5cm}

Il coefficiente di desaturazione è stato trovato con una taratura manuale e fissato a $k_{desat}=2$. Più è alto e migliore è l'effetto di de-saturazione, ma amplifica di più anche eventuali disturbi non reiettati dal controllo. Anche qui quindi si è scelto un trade-off.

\vspace{0.5cm}

Facciamo delle simulazioni con e senza correzione sul wind-up e confrontiamole.

\begin{figure}[h!]
    \centering
    \begin{minipage}{0.48\textwidth}
        \centering
        \includegraphics[width=\linewidth]{Immagini/PlotMatlab/controlloConfrontoWindUp.png}
        \caption{Confronto segnale di controllo}
        \label{fig:controlloConfontoWU}
    \end{minipage}\hfill
    \begin{minipage}{0.48\textwidth}
        \centering
        \includegraphics[width=\linewidth]{Immagini/PlotMatlab/uscitaConfrontoWindUp.png}
        \caption{Confronto segnale d'uscita}
        \label{fig:uscitaConfontoWU}
    \end{minipage}
\end{figure}

\vspace{0.5cm}

Si osserva che, compensando il fenomeno del \textit{wind-up}, aumentano notevolmente le prestazioni a ciclo chiuso:
\begin{itemize}
    \item diminuzione della sovraelongazione;
    \item diminuzione del tempo di assestamento.
\end{itemize}

\vspace{0.5cm}

\subsection{Feed Forward}

Introducendo la parte integrativa è stato garantito l'astatismo a gradino. Dalla teoria dei sistemi, sappiamo che un sistema di questo tipo segue la rampa con errore finito.

Per seguire la rampa con errore nullo, introduciamo il controllo in Feed Forward, azione in avanti che lega il controllo direttamente al riferimento, senza passare per l'errore.

In particolare, prende il riferimento di velocità desiderato e lo riporta, tramite un puro guadagno $K_{ff}$ come contributo al controllo, come mostrato in Figura~\ref{fig: FFSimulink}. 

\vspace{0.5cm}

\begin{figure}[h!]
    \centering
    \includegraphics[scale=0.6]{Immagini/Altre/feedForwardSimulink.PNG}
    \caption{Azione Feed Forward in Simulink}
    \label{fig: FFSimulink}
\end{figure}

\vspace{0.5cm}

Dalla teoria, il guadagno $K_{ff}$ è calcolato come segue:
\[
    k_{ff} = \frac{k_v}{k_{amp} \cdot k_{dt} \cdot k_{deamp}}=0.98
\]

\vspace{0.5cm}

Come si può notare dalla Figura\ref{fig: rispostaArampa}, il sistema adesso segue con errore nullo a regime il riferimento a rampa.

\vspace{0.5cm}
\begin{figure}[h!]
    \centering
    \includegraphics[scale = 0.60]{Immagini/PlotMatlab/rispostaRampaFF.png}
    \caption{Risposta a rampa, confronto}
    \label{fig: rispostaArampa}
\end{figure}

\newpage


\section{Modellistica degli ingressi}

Dopo aver progettato il controllore, ci occupiamo di modellare gli ingressi al sistema:

\begin{itemize}
    \item disturbo in coppia
    \item riferimento di velocità
\end{itemize}

\subsection{Disturbo: coppia di carico}

Fin'ora abbiamo tenuto il nostro motore a vuoto, ovvero senza carico (eccetto la struttura meccanica, la cui inerzia però è stata inglobata in quella del motore).

Modelliamo la presenza di una persona sul nastro con una coppia di carico, in prima approssimazione come una coppia di attrito generata tra il tappeto e il rullo.

\vspace{0.3cm}

Nell'approssimazione più semplice, per una massa m che sale sul tapis roulant, coefficiente d'attrito $\mu$:

\vspace{0.3cm}

\[
    F_{\text{attr}} = m_{\text{carico}} \cdot g \cdot \mu
\]
\[
    M_{\text{attr}} = F_{\text{attr}} \cdot r_{\text{rotazionale}}
\]

\vspace{0.1cm}

Ad esempio, con una massa di 120 kg otteniamo una coppia di carico di $Q_l = 1.47\,Nm$. Vediamo in Figura~\ref{fig: disturbo} la risposta del sistema a ingresso gradino + rampa, con e senza tale disturbo, modellato come un gradino.

Siccome l'attrito viene generato solo dalla partenza, la coppia di carico avrà lo step time in corrispondenza dell'avvio del moto.

\vspace{0.5cm}

\begin{figure}[h!]
    \centering
    \includegraphics[scale = 0.65]{Immagini/PlotMatlab/rispostaCoppiaCarico.png}
    \caption{Risposta al disturbo}
    \label{fig: disturbo}
\end{figure}

Nonostante il disturbo, il motore segue il riferimento con lieve, ma inevitabile degrado delle prestazioni.


\subsection{Riferimento: legge dei moti}

Quando un utilizzatore sceglie una data velocità di funzionamento del tapis roulant, è chiaro che non possiamo dare al sistema un riferimento a gradino, in quanto il tappeto avrebbe uno sbalzo improvviso. Il comfort d'utilizzo prevede che la velocità di regime si raggiunga dolcemente e senza brusche accelerazioni.

\vspace{0.3cm}

Genereremo pertanto dei riferimenti di velocità morbidi da accelerazione a forma triangolare.

Le scelte progettuali, su base empirica, saranno le seguenti:

\begin{itemize}
    \item Il tempo di raggiungimento della velocità desiderata sarà ottenuto da: 
    \[
    T_{\text{nec}} : \omega_{\text{fin}} = \Delta t_{\text{nec, max}} : \omega_{\text{max}}
    \]
    
    \item La metà di tale intervallo di tempo sarà dedicato alla salita dell'accelerazione e la restante metà alla discesa.
    
    \item Con tale forma geometrica prevista per $a_{\text{max}}$, essendo la velocità finale (o comunque la variazione di velocità richiesta) l'integrale dell'accelerazione, si può trovare banalmente che:
    \[
    \omega_{\text{fin}} = \frac{1}{2} a_{\text{max}} \cdot \left( \frac{T_{\text{nec}}}{2} \right)
    \quad \Rightarrow \quad 
    a_{\text{max}} = \frac{4 \cdot \omega_{\text{fin}}}{T_{\text{nec}}}
    \]
\end{itemize}

In Figure\ref{fig: leggiDIMoto}-\ref{fig: rispostaMorbida} è stata simulata una partenza da fermo fino a una velocità $v_{fin}=12\,km/h\Rightarrow w_{fin}=133.3\,rad/s$ scelta dall’utente.

\vspace{0.5cm}


\begin{figure}[h!]
    \centering
    \begin{minipage}{0.48\textwidth}
        \centering
        \includegraphics[scale = 0.5]{Immagini/PlotMatlab/leggiMotoDaFermo.png}
        \caption{Riferimento, da leggi di moto}
        \label{fig: leggiDIMoto}
    \end{minipage}\hfill
    \begin{minipage}{0.48\textwidth}
        \centering
        \includegraphics[scale = 0.5]{Immagini/PlotMatlab/rispostaLeggiMoto.png}
        \caption{Risposta morbida}
        \label{fig: rispostaMorbida}
    \end{minipage}
\end{figure}


\vspace{0.5cm}

Come si vede, la risposta del motore segue il riferimento, pur con un piccolissimo errore (praticamente inavvertibile) durante il tratto in parabola.

\vspace{0.3cm}

Si osservi che tali leggi di moto sono facilmente generalizzabili al caso in cui si parta da qualsiasi velocità. L'integrale dell'accelerazione rappresenterà in quel caso la variazione della velocità a partire da un certo $v_0$.

In Figura \ref{fig: decelazione} ad esempio, è riportata la generazione di una decelerazione da $v_0=12\,km/h \Rightarrow w_{0}=133.3\,rad/s $ a velocità $v_{fin}=5\,km/h\Rightarrow w_{fin}=55.6\,rad/s$.

\vspace{0.5cm}

\begin{figure}[h!]
    \centering
    \includegraphics[scale = 0.5]{Immagini/PlotMatlab/leggiMotoDecelerazione.png}
    \caption{Riferimento in decelerazione}
    \label{fig: decelazione}
\end{figure}

\vspace{0.5cm}

Daremo per scontato che il microcontrollore sappia generare tali riferimenti con appositi algoritmi.

\newpage


\section{Conclusioni}

Il progetto ha avuto come obiettivo lo studio, la modellazione e il controllo di un sistema di trazione per tapis roulant basato su motore in corrente continua. Dopo un’accurata selezione del motore e la verifica delle sue prestazioni rispetto ai requisiti meccanici del sistema, è stato modellato e simulato in ambiente \textit{Simulink}.

L'adozione di una logica di controllo PI, con integrazione delle tecniche di anti-windup e feedforward, ha permesso di ottenere un sistema stabile e conforme alle specifiche richieste: ripple di corrente contenuto, sovraelongazione contenuta, risposta fluida e robusta ai disturbi.

È stata inoltre sviluppata una logica di generazione del riferimento in velocità a forma triangolare, per garantire comfort e sicurezza d'uso durante il funzionamento del tapis roulant.

\bigskip

Tra gli sviluppi futuri si potrebbero considerare:

\begin{itemize}
    \item lo studio approfondito di disturbi, attriti e parametri del motore al fine di avere una modellistica più coerente con la realtà;
    \item la scelta di un microcontrollore adatto allo scopo e la scrittura dell'algoritmo di controllo e generazione di riferimenti; 
    \item l'utilizzo di tecniche di controllo alternative al classico controllo PI (ad esempio tecniche di controllo robusto per abbattere il peso delle numerose incertezze);
    \item l’implementazione e il testing su su hardware reale con feedback da sensori fisici.
\end{itemize}

\bigskip


\begin{figure}[h!]
    \centering
    \includegraphics[scale = 0.5]{Immagini/Altre/regolatoreFinale.PNG}
    \caption{Schema finale}
    \label{fig: modelloFinale}
\end{figure}

\vspace{0.5cm}


\newpage

\section*{Riferimenti}

Il progetto completo, comprensivo di script \textsc{Matlab} e modello \textsc{Simulink} per la simulazione del sistema, è disponibile al seguente link:

\vspace{0.5cm}

\url{https://github.com/francesco1203/DC_motor_drive}



\end{document}
